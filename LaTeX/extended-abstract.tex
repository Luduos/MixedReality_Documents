\documentclass{sigchi-ext}
% Please be sure that you have the dependencies (i.e., additional
% LaTeX packages) to compile this example.
\usepackage[T1]{fontenc}
\usepackage{textcomp}
\usepackage[scaled=.92]{helvet} % for proper fonts
\usepackage{graphicx} % for EPS use the graphics package instead
\usepackage{balance}  % for useful for balancing the last columns
\usepackage{booktabs} % for pretty table rules
\usepackage{ccicons}  % for Creative Commons citation icons
\usepackage{ragged2e} % for tighter hyphenation

% Some optional stuff you might like/need.
% \usepackage{marginnote} 
% \usepackage[shortlabels]{enumitem}
% \usepackage{paralist}
 \usepackage[utf8]{inputenc} % for a UTF8 editor only


%% EXAMPLE BEGIN -- HOW TO OVERRIDE THE DEFAULT COPYRIGHT STRIP --
%\copyrightinfo{test}
%% EXAMPLE END

% Paper metadata (use plain text, for PDF inclusion and later
% re-using, if desired).  Use \emtpyauthor when submitting for review
% so you remain anonymous.
\def\plaintitle{Concept of the final Mixed Reality WS2017/2018 project: Dragonhood} \def\plainauthor{Akash Joseph Castelino, David Liebemann}
\def\emptyauthor{}
\def\plainkeywords{Concept; Mixed Reality; Augmented Reality; Interactive game; }
\def\plaingeneralterms{Concept, Documentation}

\title{Concept of the final Mixed Reality WS2017/2018 project: Dragonhood}

\numberofauthors{2}
% Notice how author names are alternately typesetted to appear ordered
% in 2-column format; i.e., the first 4 autors on the first column and
% the other 4 auhors on the second column. Actually, it's up to you to
% strictly adhere to this author notation.
\author{%
  \alignauthor{%
    \textbf{Akash Castelino}\\
    \affaddr{Saarland University} \\
    \email{s8akcast@stud.uni-saarland.de} 
}
\alignauthor{%
    \textbf{David Liebemann}\\
    \affaddr{Saarland University}\\
    \email{s8dalieb@stud.uni-saarland.de} 
} 
}

% Make sure hyperref comes last of your loaded packages, to give it a
% fighting chance of not being over-written, since its job is to
% redefine many LaTeX commands.
\definecolor{linkColor}{RGB}{6,125,233}
\hypersetup{%
  pdftitle={\plaintitle},
%  pdfauthor={\plainauthor},
  pdfauthor={\emptyauthor},
  pdfkeywords={\plainkeywords},
  bookmarksnumbered,
  pdfstartview={FitH},
  colorlinks,
  citecolor=black,
  filecolor=black,
  linkcolor=black,
  urlcolor=linkColor,
  breaklinks=true,
  draft
}

% \reversemarginpar%

\begin{document}

%% For the camera ready, use the commands provided by the ACM in the Permission Release Form.
\CopyrightYear{2017}

%\set{rightsretained}
\conferenceinfo{Saarbrücken}{Saarland University}
\isbn{todo}
\doi{todo}
%% Then override the default copyright message with the \acmcopyright command.
\copyrightinfo{\acmcopyright}

\maketitle

% Uncomment to disable hyphenation (not recommended)
% https://twitter.com/anjirokhan/status/546046683331973120
\RaggedRight{} 

% Do not change the page size or page settings.
\begin{abstract}
  UPDATED---\today. This sample paper describes the formatting
  requirements for SIGCHI Extended Abstract Format, and this sample
  file offers recommendations on writing for the worldwide SIGCHI
  readership. Please review this document even if you have submitted
  to SIGCHI conferences before, as some format details have changed
  relative to previous years. Abstracts should be about 150
  words. Required.
\end{abstract}

\keywords{\plainkeywords}

\category{H.5.m}{Information interfaces and presentation (e.g.,
  HCI)}{Miscellaneous}\category{See}{\url{http://acm.org/about/class/1998/}}{for
  full list of ACM classifiers. This section is required.}

\section{Introduction}

Using Unity3D \cite{unity3d}.
To improve readability, we will 

\section{Choice between Creation Mode and Play Mode}

In the beginning the player will be presented with a Menu Screen in which he is able to choose between Creation Mode and Play Mode.

\section{Creation Mode} 
\label{sec:Creation}

The Creation Mode lets players adjust an adventure to their neighbourhood and save those adjustments, to later experience the adventure during Play Mode. Those adjustments include the marking of two locations on a real world map. Creation Mode is an addition to the original prototype.

\subsection{Entering Creation Mode}

Upon entering Creation Mode the player will be presented with a map depicting the real world, centred on the current GPS-position of the players phone. Map textures will be downloaded from Google Maps \cite{googlemaps} using the Google Maps Developer API \cite{googlemapsAPI}.

The player position and view direction will be represented by a coloured pointer. At this state of adventure-creation the UI will show buttons enabling the player to set markers, zoom in and out of the map or exit the Creation Mode.

By holding the button labeled ``Zoom'' and tilting the phone, players will be able to zoom in and out of the map.

A possible scenario upon entering the Creation Mode can be seen in figure \ref{fig:CM_Entering}.

\begin{marginfigure}
  \begin{minipage}{\marginparwidth}
    \centering
    \includegraphics[width=0.9\marginparwidth]{figures/CM_Entering}
    \caption{Sample view upon entering the Creation Mode.}~\label{fig:CM_Entering}
  \end{minipage}
\end{marginfigure}
 

\subsection{Setting a marker}

test

\subsection{Saving the adventure}
What happens first?
UI?
After first marker?
After second marker?
Saving?
Closing?

\subsection{Exiting Creation Mode}

\section{Play Mode}
\label{sec:Play}

\subsection{Class selection}

How will it be started?
How will a selection be represented?
How will classes differ during search for 

\subsection{Puzzlebox}
Introduction.




%\marginpar{%
%  \vspace{-45pt} \fbox{%
%   \begin{minipage}{0.925\marginparwidth}
%      \textbf{Good Utilization of the Side Bar} \\
%      \vspace{1pc} \textbf{Preparation:} Do not change the margin
%      dimensions and do not flow the margin text to the
%      next page. \\
%      \vspace{1pc} \textbf{Materials:} The margin box must not intrude
%      or overflow into the header or the footer, or the gutter space
%      between the margin paragraph and the main left column. The text
%      in this text box should remain the same size as the body
%      text. Use the \texttt{{\textbackslash}vspace{}} command to set
%      the margin
%      note's position. \\
%      \vspace{1pc} \textbf{Images \& Figures:} Practically anything
%      can be put in the margin if it fits. Use the
%      \texttt{{\textbackslash}marginparwidth} constant to set the
%      width of the figure, table, minipage, or whatever you are trying
%      to fit in this skinny space.
%    \end{minipage}}\label{sec:sidebar} }






\begin{itemize}\compresslist%
\item Write in a straightforward style. Use simple sentence
  structure. Try to avoid long sentences and complex sentence
  structures. Use semicolons carefully.

\end{itemize}

% \begin{figure}
%   \includegraphics[width=.9\columnwidth]{figures/ea-figure2}
%   \caption{If your figure has a light background, you can set its
%     outline to light gray, like this, to make a box around
%     it.}\label{fig:bats}
% \end{figure}

%\begin{marginfigure}[-35pc]
%  \begin{minipage}{\marginparwidth}
%    \centering
%    \includegraphics[width=0.9\marginparwidth]{figures/cats}
%    \caption{In this image, the cats are tessellated within a square
%      frame. Images should also have captions and be within the
%      boundaries of the sidebar on page~\pageref{sec:sidebar}. Photo:
%      \cczero~jofish on Flickr.}~\label{fig:marginfig}
%  \end{minipage}
%\end{marginfigure}

%\begin{figure*}
%  \centering
%  \includegraphics[width=1.3\columnwidth]{figures/map}
%  \caption{In this image, the map maximizes use of space. You can make
%    figures as wide as you need, up to a maximum of the full width of
%    both columns. Note that \LaTeX\ tends to render large figures on a
%    dedicated page. Image: \ccbynd~ayman on Flickr.}~\label{fig:cats}
%\end{figure*}



\balance{} 

\bibliographystyle{SIGCHI-Reference-Format}
\bibliography{sample}

\end{document}

%%% Local Variables:
%%% mode: latex
%%% TeX-master: t
%%% End:
