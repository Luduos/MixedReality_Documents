\documentclass{sigchi-ext}
% Please be sure that you have the dependencies (i.e., additional
% LaTeX packages) to compile this example.
\usepackage[T1]{fontenc}
\usepackage{textcomp}
\usepackage[scaled=.92]{helvet} % for proper fonts
\usepackage{graphicx} % for EPS use the graphics package instead
\usepackage{balance}  % for useful for balancing the last columns
\usepackage{booktabs} % for pretty table rules
\usepackage{ccicons}  % for Creative Commons citation icons
\usepackage{ragged2e} % for tighter hyphenation

% Some optional stuff you might like/need.
% \usepackage{marginnote} 
% \usepackage[shortlabels]{enumitem}
% \usepackage{paralist}
 \usepackage[utf8]{inputenc} % for a UTF8 editor only


%% EXAMPLE BEGIN -- HOW TO OVERRIDE THE DEFAULT COPYRIGHT STRIP --
%\copyrightinfo{test}
%% EXAMPLE END

% Paper metadata (use plain text, for PDF inclusion and later
% re-using, if desired).  Use \emtpyauthor when submitting for review
% so you remain anonymous.
\def\plaintitle{Concept of the final Mixed Reality WS2017/2018 project: Point-and-Click adventure} \def\plainauthor{Akash Joseph Castelino, David Liebemann}
\def\emptyauthor{}
\def\plainkeywords{Concept; Mixed Reality; Augmented Reality; Interactive game; }
\def\plaingeneralterms{Concept, Documentation}

\title{Concept of the final Mixed Reality WS2017/2018 project: Point-and-Click adventure}

\numberofauthors{2}
% Notice how author names are alternately typesetted to appear ordered
% in 2-column format; i.e., the first 4 autors on the first column and
% the other 4 auhors on the second column. Actually, it's up to you to
% strictly adhere to this author notation.
\author{%
  \alignauthor{%
    \textbf{Akash Castelino}\\
    \affaddr{Saarland University} \\
    \email{s8akcast@stud.uni-saarland.de} 
}
\alignauthor{%
    \textbf{David Liebemann}\\
    \affaddr{Saarland University}\\
    \email{s8dalieb@stud.uni-saarland.de} 
} 
}

% Make sure hyperref comes last of your loaded packages, to give it a
% fighting chance of not being over-written, since its job is to
% redefine many LaTeX commands.
\definecolor{linkColor}{RGB}{6,125,233}
\hypersetup{%
  pdftitle={\plaintitle},
%  pdfauthor={\plainauthor},
  pdfauthor={\emptyauthor},
  pdfkeywords={\plainkeywords},
  bookmarksnumbered,
  pdfstartview={FitH},
  colorlinks,
  citecolor=black,
  filecolor=black,
  linkcolor=black,
  urlcolor=linkColor,
  breaklinks=true,
  draft
}

% \reversemarginpar%

\begin{document}

%% For the camera ready, use the commands provided by the ACM in the Permission Release Form.
\CopyrightYear{2018}

%\set{rightsretained}
\conferenceinfo{Saarbrücken}{Saarland University}
\isbn{Unknown}
\doi{Unknown}
%% Then override the default copyright message with the \acmcopyright command.
\copyrightinfo{\acmcopyright}

\maketitle

% Uncomment to disable hyphenation (not recommended)
% https://twitter.com/anjirokhan/status/546046683331973120
\RaggedRight{} 

% Do not change the page size or page settings.
\begin{abstract}
  This paper describes the concept for the final project of the seminar ``Making Mixed Reality great again!''. In the beginning, the project and required terminology are introduced, as well as requirements for starting the proposed Augmented Reality, locally cooperative game. 
  
  The following sections lead through the two game modes. The section ``Creation Mode'' describes the creative aspect of the game, by explaining how players can create and save their own adventures on basis of their real world surroundings.
  
  The section ``Play Mode'' describes how players can load up previously created adventures and connect with friends via Wi-Fi. Players will have to decide on a certain role they have to fit in, and navigate towards certain tasks they have to fulfil in order to complete the adventure.
  
  The last two sections cover the details of the two scenarios -- certain tasks -- players have to solve.
\end{abstract}

\keywords{\plainkeywords}

\category{H.5.m}{Information interfaces and presentation (e.g.,
  HCI)}{Miscellaneous}\category{See}{\url{http://acm.org/about/class/1998/}}{for
  full list of ACM classifiers. This section is required.}

\section{Introduction}
\subsection{Previous project}
In a previous micro-project, a mobile application was developed using Multi-Touch Input, Mobile Display Output and a setting involving the creation of a ``Point and Click Neighbourhood adventure''. This game-prototype focused on the settings gaming aspect and included a cooperative scenario called ``Puzzle-box'' and a single player scenario called ``Dragon''. In this paper, a scenario equates to a certain task a player has to pass in order to gain a reward. Scenarios were started by scanning a so called ``Vuforia-markers'' -- certain images printed on a piece of paper. 

Vuforia is an Augmented Reality Software Development Kit for mobile devices that enables the creation of Augmented Reality applications. It uses Computer Vision technology to recognize and track planar images in real-time. \cite{vuforiaExplanation}

A scenario was started by pointing the mobile phones camera towards the Vuforia-marker, which would lead Vuforia to display the scenario on the screen. Scenarios were solved by accomplishing a number of win-conditions. All aspects of finding and solving every scenario will be further referred to as an ``adventure''. 

\begin{marginfigure}[-15pc]
	\centering
	\begin{minipage}{\marginparwidth}
		\centering
		\includegraphics[width=1\marginparwidth]{figures/Vuforia_Questionmarks}
		\caption{Vuforia-marker for the scenario ``Puzzle-box''.}\label{fig:Vuforia_Questionmarks}
	\end{minipage}
	\begin{minipage}{\marginparwidth}
		\centering
		\includegraphics[width=1\marginparwidth]{figures/Vuforia_Dragon}
		\caption{Vuforia-marker for the scenario ``Dragon''.}\label{fig:Vuforia_Dragon}
	\end{minipage}
\end{marginfigure}

\subsection{Final project}

Our Final Project concept is an extension of this ``Point and Click Neighbourhood adventure'' using Multi-Touch Input and Mobile Display Output. The additional creative aspect allows the user to place the ``Puzzle-box'' and ``Dragon'' scenarios anywhere and  mark their GPS-positions. The extended playability aspect includes mandatory, cooperative gameplay to win against the Dragon. Further additions include the use of ``Mobile sensor input'' via Accelerometer to perform actions, and a new inventory item.

The project will be implemented using Unity3D \cite{unity3d}, which is a cross-platform 3D engine mainly used for video game development. 

\subsection{Motivation}

In modern day life, especially in metropolitan areas, people often don't know the names of their neighbours or have never explored their neighbourhood. It is also harder for parents to motivate their children to go outside and play, if all they really want is to stay at home and play games on their phones. On the other hand, if children want to explore the neighbourhood, parents might want to be able to guide them away from dangerous areas.

This game motivates players to explore their surroundings while looking for good spots to create scenario locations during Creation Mode or while searching for scenarios during Play Mode. It demands cooperative play in order to accomplish tasks, so players will have to find people nearby to play the game with. To successfully finish an adventures, players will also have to speak with each in the real world, teaching team play and communication skills. Parents will be able to create adventures for their children, to either have some fun family time or to keep children away from dangerous locations.

Previously, the game could only be played when the users knew exactly where the Vuforia-markers are located. The new features make the game playable in a more exploratory fashion, leading to a mix of physical movement while searching for scenarios along with virtual engagement on the mobile device. 

\subsection{Possible Problems}

Players choosing to play the adventure will need to rely on the creator to select the correct Vuforia-marker and place it at the chosen GPS-location in a way, that the Vuforia-marker will be difficult to remove, but easy to access. As it is in the best interest of everyone playing the game, we will assume that creators act with caution while creating an adventure. Experiences with the Geocaching\textsuperscript{\textregistered} Mobile App \cite{app:geocaching} support these assumptions.

\subsection{Player-side requirements}

The application will be usable on mobile phones running Android: OS 4.4 or higher or iOS 7.0 or higher \cite{unityRequirements}. The phone will need certain sensors and features for the application to work correctly:

\begin{itemize}\compresslist%
	\item Gyroscope
	\item Accelerometer
	\item GPS
	\item Camera
	\item Mobile Hotspot creation \cite{desc:hotspot}
\end{itemize}

\section{Choice between Creation Mode and Play Mode}

Upon starting the application, players will be presented with a menu screen in which they are able to choose between Creation Mode and Play Mode.


\section{Creation Mode} 
\label{sec:Creation}

The Creation Mode allows players to adjust an adventure to their neighbourhood and save those adjustments. This currently only includes the marking of two scenario-locations on a real world map, but offers possible extensions for later development cycles. Creation Mode is an addition to the original prototype.

\subsection{Entering Creation Mode}

Upon entering Creation Mode the player will be presented with a map depicting the real world, centred on the current GPS-position of the players phone. Map textures will be downloaded from Google Maps \cite{googlemaps} using the Google Maps Developer API \cite{googlemapsAPI}.

The player position and view direction will be represented by a coloured pointer. At this state of adventure-creation the UI will show buttons enabling the player to set markers, zoom in and out of the map or exit the Creation Mode.

By holding the button labelled ``Zoom'' and tilting the phone, players will be able to zoom in and out of the map.

A possible scene upon entering the Creation Mode can be seen in figure \ref{fig:CM_Entering}.
 

\subsection{Setting Scenario-markers}

\begin{figure}
	\includegraphics[width=1\columnwidth]{figures/CM_Markers}
	\caption{Sample view, after the player set two scenario-markers.}\label{fig:CM_Markers}
\end{figure}

Markers display GPS-locations of scenarios in the real world. They will be represented by coloured points on the map. Upon entering the Creation Mode there will be no markers on the map. Players will have to determine scenario-locations by moving to a point in the real world and selecting the button ``Set Marker''.

\begin{marginfigure}[8pc]
	\begin{minipage}{\marginparwidth}
		\centering
		\includegraphics[width=0.9\marginparwidth]{figures/CM_Entering}
		\caption{Sample view upon entering the Creation Mode.}~\label{fig:CM_Entering}
	\end{minipage}
\end{marginfigure}

After selection, players will be presented with the choice between setting two different markers:
\begin{itemize}\compresslist%
	\item Puzzle-master: Puzzle-box
	\item Fighter: Dragon
\end{itemize}
The list is written in a $Player-class: Scenario-name$ representation, displaying the current possible scenarios for each class. Player-class and scenarios are described in section ``Play Mode'' on page \pageref{sec:Play}.

After selecting a scenario, a marker will appear on the current GPS-location of the player. Players will now have to select the corresponding Vuforia-marker and place it at the chosen real world position.

Players will need to set both a ``Puzzle-box'' and a ``Dragon'' marker to be able to save the current adventure. It will not be possible to set two markers of the same type or save the current adventure with less than two markers on the map.

An example-view of the display after a player set both markers is depicted in figure \ref{fig:CM_Markers}.

\subsection{Saving the adventure}

After both markers have been set by the player, a ``Save adventure'' button will become available. Pressing this button enables the player to store the current positions of the markers and assign a name to the created adventure. After completing this process, the adventure will be available to play during Play Mode.

\subsection{Exiting Creation Mode}

Players can choose to exit Creation Mode at any time. Unsaved progress during creation will be lost upon exiting, only saved adventures will be available to play.

To create a new adventure, players will need to exit and re-enter Creation Mode.


\subsection{Possible extension}

An idea to improve both on the creational and the navigational aspects of the application is to split up the Vuforia markers into multiple puzzle pieces. To solve a scenario, players would have to find all pieces belonging to the corresponding marker. Creators of adventures would have to be able to define the amount of pieces belonging to one marker and to mark each one of them accordingly on the map.

This feature will be optionally implemented.

\section{Play Mode}
\label{sec:Play}


The Play Mode allows players to experience previously created adventures. They will need to work cooperatively in a group of at least two people to successfully complete an adventure. To achieve this, players will have to navigate towards the GPS-locations of each scenario in the real world and then complete both scenarios. 

Play Mode was already present in the original prototype, but will be extended to incorporate the Creation Mode and to support cooperative play in both scenarios.

\subsection{Entering Play Mode}

Upon entering Play Mode, the player will be presented with the current video input of the phones camera. He will additionally be presented with a range of UI-elements, as depicted in figure \ref{fig:PM_Entering}.

\begin{marginfigure}[5pc]
	\begin{minipage}{\marginparwidth}
		\centering
		\includegraphics[width=1\marginparwidth]{figures/PM_Entering}
		\caption{Sample view of Play Mode. We can see the background showing the video input, the available buttons if playing as Host and the connection status (1), the current Player-class (2) and whether or not the player has a feather in the inventory (3). }~\label{fig:PM_Entering}
	\end{minipage}
\end{marginfigure}

\subsection{Connecting players}
To play cooperatively, players will have to connect their applications over a Wi-fi network. All players need to have the current version of the application running on their mobile phones. One player will start a Software Hotspot \cite{desc:hotspot}, the other players then connect to that hotspot.

Once everybody is connected, one player will need to press the ``Host'' button and will subsequently be denoted as Host. Every player connected to the hotspot will automatically connect to the Host in the application. 

The Host can now choose between starting the class selection and loading an adventure by pressing the corresponding buttons.



\subsection{Class selection}

After a click on the ``Class'' button, every player will be presented with the choice between one of the two following classes:
\begin{itemize}\compresslist%
	\item \textit{Puzzle-master:} Required to solve the Puzzle-box. Can locate the position of the Puzzle-box. Can carry a bell.
	\item \textit{Fighter:} Required to win against the Dragon. Can locate the position of the Dragon. Is able to carry a feather.
\end{itemize}

Each class will have to be picked by at least one player in order to complete the adventure. The choice of class will be displayed in the top right corner of the screen, as depicted in figure \ref{fig:PM_Entering}, number 2.

\begin{figure}
	\centering
	\begin{minipage}{.5\columnwidth}
		\centering
		\includegraphics[width=1\linewidth]{figures/Class_Fighter}
		\captionof{figure}{Symbol of the Fighter class}
		\label{fig:Class_Fighter}
	\end{minipage}%
	\begin{minipage}{.5\columnwidth}
		\centering
		\includegraphics[width=1\linewidth]{figures/Class_Puzzlemaster}
		\captionof{figure}{Symbol of the Puzzle-master class}
		\label{fig:Class_Fighter}
	\end{minipage}
\end{figure}


\subsection{Loading an adventure}

After pressing the ``Load'' button, the Host will be presented with a list of the created adventures. After choosing one, the GPS-locations of the scenarios will be distributed to each connected player and the navigation ability of each class will be activated.

\subsection{Finding a scenario}

The navigation ability is represented as an arrow which can be seen as an Augmented Reality object on the players screen. The arrow will point towards the one scenario, which the players class can locate - f.e. if a player is a member of the Fighter class, his arrow will point towards the dragon scenario.

Alternatively, other navigation abilities will be implemented, depending on the possible design decisions for the Creation Mode and on the remaining development time. 

If the decision to split up Vuforia markers into multiple puzzle pieces is made, this will entail the implementation of a real world map with the different pieces represented as points of interest. In this case the navigation via Augemented Reality arrows would not be viable anymore. This navigation mode would be similar to the view in figure \ref{fig:CM_Markers}.

Additionally, to cope with possible inprecisions of GPS-navigation, a ``hot-and-cold'' mode could be implemented. The application would give feedback about the rough proximity of nearby markers, instead of pointing toward the exact, but possibly positionally off location.

\subsection{Scanning a scenario}

Players can scan and thus activate a scenario, by pointing their camera towards the Vuforia-marker. Once the application has identified the marker, the corresponding scenario will be shown as an Augmented Reality object on top of the marker, as depicted in figures \ref{fig:AR_Box1} and \ref{fig:AR_Dragon1}. Players can then start solving the scenarios.

\begin{figure}
	\centering
	\includegraphics[width=1\columnwidth]{figures/PM_AR_Box}
	\caption{Sample view of the Puzzle-box as an Augmented Reality object.}\label{fig:AR_Box1}
\end{figure}

\begin{figure}
	\centering
	\includegraphics[width=1\columnwidth]{figures/PM_AR_Dragon}
	\caption{Sample view of the Dragon as an Augmented Reality object.}\label{fig:AR_Dragon1}
\end{figure}

\section{Puzzle-box}

The Puzzle-box is the first of two scenarios the player will have to solve in order to successfully complete an adventure.

The Puzzle-box consists of five visible sides, with different numbers depicted on each side, similar to an ordinary dice. The number on each side corresponds to the number of fingers the player has to touch it with - f.e. if a side shows the number three, the player has to touch the side on the screen with three fingers.

If the correct amount of fingers was being used, the side will turn yellow to signal a correct input and be counted as ``solved''. Otherwise it will turn red for a second and then return to the original colour, in order to signal incorrect input.

Once a side has turned yellow, the player who solved that side can not solve any other side of the box. He will need to wait for another player on the network to solve the corresponding side, which is defined as the opposite side. If the other player solves the opposite side, both sides will turn green. Otherwise all sides on the Puzzle-box will turn red and the Puzzle-box will be reset to its original state.

Players will have to work together, as the Puzzle-box can not be solved by one person alone. All players will be able to see whether or not a side has been solved, but only the Puzzle-master will be able to see the numbers depicted on the sides of the box, giving him the most important role in this scenario. The Puzzle-master will have to communicate in the real world with the Fighter in order to tell him, which number is depicted on the side the Fighter intends to solve.

\subsection{Completing the scenario}

The scenario is seen as completed, if all five sides of the Puzzle-box have turned green. Then the scenario presents the reward as an Augmented Reality object on top of the Puzzle-box: A feather and a bell, which are both required to solve the Dragon-scenario.

Players can pick up the item corresponding to their class and store it in their inventory, as depicted in figure \ref{fig:PM_Entering}, number 3.

\subsection{Improvements over the original prototype}

Improvements will include implementing better looking textures and effects, as well as an improved usability, which includes a more intuitive description of the task and a easier handling of the Multi-Touch-Controls.

\section{Dragon}

The dragon needs to be defeated by completing all actions in a certain action-sequence, which include both multi touch gestures and shake gestures. Both players will adopt specific roles based on the item they acquired from the mystery box. The player with a feather will perform the action sequence on the dragon -- examples for actions and the corresponding hints can be found in figures \ref{fig:MotionLeft} and \ref{fig:MotionRight}. Once the dragon turned towards the player who performed an action, the player with a bell has to ring it in order to distract the dragon.

\begin{figure}
	\centering
	\begin{minipage}{.4\columnwidth}
		
		\includegraphics[width=1\linewidth]{figures/PM_UpDown}
		\captionof{figure}{An action the Fighter has to perform during the fight with the dragon. Players have to swipe up and down repeatedly. }
		\label{fig:MotionLeft}
	\end{minipage}%
	\begin{minipage}{.4\columnwidth}
		
		\includegraphics[width=1\linewidth]{figures/PM_MotionLeft}
		\captionof{figure}{Players have to swipe to the left with three fingers.}
		\label{fig:MotionRight}
	\end{minipage}
\end{figure}

Gameplay key points include:

\begin{itemize}\compresslist%
\item The multi-touch action sequence on the dragon can only be performed from behind the dragon. Players have to walk around the Vuforia-marker to achieve this.
\item If actions are performed incorrectly or if the action sequence is attempted in front of the dragon, it will get visibly angry, indicated by an animation of the dragon-model.
\item Ringing the bell by shaking the phone will make the dragon look at the player with the bell.
\item The dragon will always follow one player at any time.
\item Performing an action correctly will make the dragon look at the player who performed the action.
\item Performing all actions correctly will make the dragon fly away from the neighbourhood.
\end{itemize}


\balance{} 

\bibliographystyle{SIGCHI-Reference-Format}
\bibliography{MR_Concept}

\end{document}

%%% Local Variables:
%%% mode: latex
%%% TeX-master: t
%%% End:
