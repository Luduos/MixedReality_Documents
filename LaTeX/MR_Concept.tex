\documentclass{sigchi-ext}
% Please be sure that you have the dependencies (i.e., additional
% LaTeX packages) to compile this example.
\usepackage[T1]{fontenc}
\usepackage{textcomp}
\usepackage[scaled=.92]{helvet} % for proper fonts
\usepackage{graphicx} % for EPS use the graphics package instead
\usepackage{balance}  % for useful for balancing the last columns
\usepackage{booktabs} % for pretty table rules
\usepackage{ccicons}  % for Creative Commons citation icons
\usepackage{ragged2e} % for tighter hyphenation

% Some optional stuff you might like/need.
% \usepackage{marginnote} 
% \usepackage[shortlabels]{enumitem}
% \usepackage{paralist}
 \usepackage[utf8]{inputenc} % for a UTF8 editor only


%% EXAMPLE BEGIN -- HOW TO OVERRIDE THE DEFAULT COPYRIGHT STRIP --
%\copyrightinfo{test}
%% EXAMPLE END

% Paper metadata (use plain text, for PDF inclusion and later
% re-using, if desired).  Use \emtpyauthor when submitting for review
% so you remain anonymous.
\def\plaintitle{Concept of the final Mixed Reality WS2017/2018 project: Dragonhood} \def\plainauthor{Akash Joseph Castelino, David Liebemann}
\def\emptyauthor{}
\def\plainkeywords{Concept; Mixed Reality; Augmented Reality; Interactive game; }
\def\plaingeneralterms{Concept, Documentation}

\title{Concept of the final Mixed Reality WS2017/2018 project: Dragonhood}

\numberofauthors{2}
% Notice how author names are alternately typesetted to appear ordered
% in 2-column format; i.e., the first 4 autors on the first column and
% the other 4 auhors on the second column. Actually, it's up to you to
% strictly adhere to this author notation.
\author{%
  \alignauthor{%
    \textbf{Akash Castelino}\\
    \affaddr{Saarland University} \\
    \email{s8akcast@stud.uni-saarland.de} 
}
\alignauthor{%
    \textbf{David Liebemann}\\
    \affaddr{Saarland University}\\
    \email{s8dalieb@stud.uni-saarland.de} 
} 
}

% Make sure hyperref comes last of your loaded packages, to give it a
% fighting chance of not being over-written, since its job is to
% redefine many LaTeX commands.
\definecolor{linkColor}{RGB}{6,125,233}
\hypersetup{%
  pdftitle={\plaintitle},
%  pdfauthor={\plainauthor},
  pdfauthor={\emptyauthor},
  pdfkeywords={\plainkeywords},
  bookmarksnumbered,
  pdfstartview={FitH},
  colorlinks,
  citecolor=black,
  filecolor=black,
  linkcolor=black,
  urlcolor=linkColor,
  breaklinks=true,
  draft
}

% \reversemarginpar%

\begin{document}

%% For the camera ready, use the commands provided by the ACM in the Permission Release Form.
\CopyrightYear{2017}

%\set{rightsretained}
\conferenceinfo{Saarbrücken}{Saarland University}
\isbn{Unknown}
\doi{Unknown}
%% Then override the default copyright message with the \acmcopyright command.
\copyrightinfo{\acmcopyright}

\maketitle

% Uncomment to disable hyphenation (not recommended)
% https://twitter.com/anjirokhan/status/546046683331973120
\RaggedRight{} 

% Do not change the page size or page settings.
\begin{abstract}
  UPDATED---\today. This sample paper describes the formatting
  requirements for SIGCHI Extended Abstract Format, and this sample
  file offers recommendations on writing for the worldwide SIGCHI
  readership. Please review this document even if you have submitted
  to SIGCHI conferences before, as some format details have changed
  relative to previous years. Abstracts should be about 150
  words. Required.
\end{abstract}

\keywords{\plainkeywords}

\category{H.5.m}{Information interfaces and presentation (e.g.,
  HCI)}{Miscellaneous}\category{See}{\url{http://acm.org/about/class/1998/}}{for
  full list of ACM classifiers. This section is required.}

\section{Introduction}

Using Unity3D \cite{unity3d}.
Explain an adventure and scenarios! What is a Vuforia-marker? 

\subsection{Possible Problems}

Players choosing to play the adventure will need to rely on the creator to select the correct Vuforia-marker and place it at the chosen GPS-location in a way that the Vuforia-marker will be difficult to remove but easy to access. As it is in the best interest of everyone playing the game, we will assume that creators will act with caution while creating an adventure. Experiences with the Geocaching\textsuperscript{\textregistered} Mobile App \cite{app:geocaching} support these assumptions.

\subsection{Requirements to play}

Which sensors and software capabilites does the phone need, what kind of OS is supported?

Gyro, GPS, Accelerometer, Camera, Android 4.4 or higher, IOS? 

\section{Choice between Creation Mode and Play Mode}

Upon starting the application, players will be presented with a menu screen in which they are able to choose between Creation Mode and Play Mode.


\section{Creation Mode} 
\label{sec:Creation}

The Creation Mode allows players to adjust an adventure to their neighbourhood and save those adjustments. This currently only includes the marking of two scenario-locations on a real world map, but offers possible extensions for later development cycles. Creation Mode is an addition to the original prototype.

\begin{marginfigure}[-20pc]
	\begin{minipage}{\marginparwidth}
		\centering
		\includegraphics[width=0.9\marginparwidth]{figures/CM_Entering}
		\caption{Sample view upon entering the Creation Mode.}~\label{fig:CM_Entering}
	\end{minipage}
\end{marginfigure}


\subsection{Entering Creation Mode}

Upon entering Creation Mode the player will be presented with a map depicting the real world, centred on the current GPS-position of the players phone. Map textures will be downloaded from Google Maps \cite{googlemaps} using the Google Maps Developer API \cite{googlemapsAPI}.

The player position and view direction will be represented by a coloured pointer. At this state of adventure-creation the UI will show buttons enabling the player to set markers, zoom in and out of the map or exit the Creation Mode.

By holding the button labelled ``Zoom'' and tilting the phone, players will be able to zoom in and out of the map.

A possible scene upon entering the Creation Mode can be seen in figure \ref{fig:CM_Entering}.


 

\subsection{Setting Scenario-markers}

\begin{figure}
	\includegraphics[width=1\columnwidth]{figures/CM_Markers}
	\caption{Sample view, after the player set two scenario-markers.}\label{fig:CM_Markers}
\end{figure}

Markers display GPS-locations of scenarios in the real world. They will be represented by coloured points on the map. Upon entering the Creation Mode there will be no markers on the map. Players will have to determine scenario-locations by moving to a point in the real world and selecting the button ``Set Marker''.

After selection, players will be presented with the choice between setting two different markers:
\begin{itemize}\compresslist%
	\item Puzzle-master: Puzzle-box
	\item Fighter: Dragon
\end{itemize}
The list is written in a $Player-class: Scenario-name$ representation, displaying the current possible scenarios for each class. Player-class and scenarios are described in section ``Play Mode'' on page \pageref{sec:Play}.

After selecting a scenario, a marker will appear on the current GPS-location of the player. Players will now have to select the corresponding Vuforia-marker and place it at the chosen real world position.

Players will need to set both a ``Puzzle-box'' and a ``Dragon'' marker to be able to save the current adventure. It will not be possible to set two markers of the same type or save the current adventure with less than two markers on the map.

An example-view of the display after a player set both markers is depicted in figure \ref{fig:CM_Markers}.



\subsection{Saving the adventure}

After both markers have been set by the player, a ``Save adventure'' button will become available. Pressing this button enables the player to store the current positions of the markers and assign a name to the created adventure. After completing this process, the adventure will be available to play during Play Mode.

\subsection{Exiting Creation Mode}

Players can choose to exit Creation Mode at any time. Unsaved progress during creation will be lost upon exiting, only saved adventures will be available to play.

To create a new adventure, players will need to exit and re-enter Creation Mode.

\begin{marginfigure}[-20pc]
	\begin{minipage}{\marginparwidth}
		\centering
		\includegraphics[width=1\marginparwidth]{figures/PM_Entering}
		\caption{Sample view of Play Mode. We can see the background showing the video input, the available buttons if playing as Host and the connection status (1), the current Player-class (2) and whether or not the player has a feather in the inventory (3). }~\label{fig:PM_Entering}
	\end{minipage}
\end{marginfigure}

\section{Play Mode}
\label{sec:Play}


The Play Mode allows players to experience previously created adventures. They will need to work cooperatively in a group of at least two people to successfully complete an adventure. To achieve this, players will have to navigate towards the GPS-locations of each scenario in the real world and then complete both scenarios. 

Play Mode was already present in the original prototype, but will be extended to incorporate the Creation Mode and to support cooperative play in both scenarios.

\subsection{Entering Play Mode}

Upon entering Play Mode, the player will be presented with the current video input of the phones camera. He will additionally be presented with a range of UI-elements, as depicted in figure \ref{fig:PM_Entering}.


\subsection{Connecting players}
To play cooperatively, players will have to connect their applications over a Wi-fi network. All players need to have the current version of the application running on their mobile phones. One player will start a Software Hotspot \cite{desc:hotspot}, the other players then connect to that hotspot.

Once everybody is connected, one player will need to press the ``Host'' button and will subsequently be denoted as Host. Every player connected to the hotspot will automatically connect to the Host in the application. 

The Host can now choose between starting the class selection and loading an adventure by pressing the corresponding buttons.



\subsection{Class selection}

After a click on the ``Class'' button, every player will be presented with the choice between one of the two following classes:
\begin{itemize}\compresslist%
	\item \textit{Puzzle-master:} Required to solve the Puzzle-box. Can locate the position of the Puzzle-box. Can not carry items.
	\item \textit{Fighter:} Required to win against the Dragon. Can locate the position of the Dragon. Is able to carry items.
\end{itemize}

Each class will have to be picked by at least one player in order to complete the adventure. The choice of class will be displayed in the top right corner of the screen, as depicted in figure \ref{fig:PM_Entering}, number 2.

\begin{figure}
	\centering
	\begin{minipage}{.5\columnwidth}
		\centering
		\includegraphics[width=1\linewidth]{figures/Class_Fighter}
		\captionof{figure}{Symbol of the Fighter class}
		\label{fig:Class_Fighter}
	\end{minipage}%
	\begin{minipage}{.5\columnwidth}
		\centering
		\includegraphics[width=1\linewidth]{figures/Class_Puzzlemaster}
		\captionof{figure}{Symbol of the Puzzle-master class}
		\label{fig:Class_Fighter}
	\end{minipage}
\end{figure}


\subsection{Loading an adventure}

After pressing the ``Load'' button, the Host will be presented with a list of the created adventures. After choosing one, the GPS-locations of the scenarios will be distributed to each connected player and the navigation ability of each class will be activated.

\subsection{Finding a scenario}

The navigation ability is represented as an arrow which can be seen as an Augmented Reality object on the players screen. The arrow will point towards the one scenario, which the players class can locate - f.e. if a player is a member of the Fighter class, his arrow will point towards the dragon scenario.

\subsection{Scanning a scenario}

Players can scan and thus activate a scenario, by pointing their camera towards the Vuforia-marker. Once the application has identified the marker, the corresponding scenario will be shown as an Augmented Reality object on top of the marker, as depicted in figures \ref{fig:AR_Box1} and \ref{fig:AR_Dragon1}. Players can then start solving the scenarios.

\begin{figure}
	\centering
	\includegraphics[width=1\columnwidth]{figures/PM_AR_Box}
	\caption{Sample view of the Puzzle-box as an Augmented Reality object.}\label{fig:AR_Box1}
\end{figure}

\begin{figure}
	\centering
	\includegraphics[width=1\columnwidth]{figures/PM_AR_Dragon}
	\caption{Sample view of the Dragon as an Augmented Reality object.}\label{fig:AR_Dragon1}
\end{figure}

\section{Puzzle-box}

The Puzzle-box is the first of two scenarios the player will have to solve in order to successfully complete an adventure.

The Puzzle-box consists of five visible sides, with different numbers depicted on each side, similar to an ordinary dice. The number on each side corresponds to the number of fingers the player has to touch it with - f.e. if a side shows the number three, the player has to touch the side on the screen with three fingers.

If the correct amount of fingers was being used, the side will turn yellow to signal a correct input and be counted as ``solved''. Otherwise it will turn red for a second and then return to the original colour, in order to signal incorrect input.

Once a side has turned yellow, the player who solved that side can not solve any other side of the box. He will need to wait for another player on the network to solve the corresponding side, which is defined as the opposite side. If the other player solves the opposite side, both sides will turn green. Otherwise all sides on the Puzzle-box will turn red and the Puzzle-box will be reset to its original state.

Players will have to work together, as the Puzzle-box can not be solved by one person alone. All players will be able to see whether or not a side has been solved, but only the Puzzle-master will be able to see the numbers depicted on the sides of the box, giving him the most important role in this scenario. The Puzzle-master will have to communicate in the real world with the Fighter in order to tell him, which number is depicted on the side the Fighter intends to solve.

\subsection{Completing the scenario}

The scenario is seen as completed, if all five sides of the Puzzle-box have turned green. Then the scenario presents the reward as an Augmented Reality object on top of the Puzzle-box: A golden feather, which is required to solve the Dragon-scenario.

Players using the Fighter class will be able to pick up the feather and store it in their inventory, as depicted in figure \ref{fig:PM_Entering}, number 3.

\subsection{Improvements over the original prototype}

Improvements will include implementing better looking textures and effects, as well as an improved usability, which includes a more intuitive description of the task and a better handling of the Multi-Touch-Controls.

\section{Dragon}

test
%\marginpar{%
%  \vspace{-45pt} \fbox{%
%   \begin{minipage}{0.925\marginparwidth}
%      \textbf{Good Utilization of the Side Bar} \\
%      \vspace{1pc} \textbf{Preparation:} Do not change the margin
%      dimensions and do not flow the margin text to the
%      next page. \\
%      \vspace{1pc} \textbf{Materials:} The margin box must not intrude
%      or overflow into the header or the footer, or the gutter space
%      between the margin paragraph and the main left column. The text
%      in this text box should remain the same size as the body
%      text. Use the \texttt{{\textbackslash}vspace{}} command to set
%      the margin
%      note's position. \\
%      \vspace{1pc} \textbf{Images \& Figures:} Practically anything
%      can be put in the margin if it fits. Use the
%      \texttt{{\textbackslash}marginparwidth} constant to set the
%      width of the figure, table, minipage, or whatever you are trying
%      to fit in this skinny space.
%    \end{minipage}}\label{sec:sidebar} }






\begin{itemize}\compresslist%
\item Write in a straightforward style. Use simple sentence
  structure. Try to avoid long sentences and complex sentence
  structures. Use semicolons carefully.

\end{itemize}

% \begin{figure}
%   \includegraphics[width=.9\columnwidth]{figures/ea-figure2}
%   \caption{If your figure has a light background, you can set its
%     outline to light gray, like this, to make a box around
%     it.}\label{fig:bats}
% \end{figure}

%\begin{marginfigure}[-35pc]
%  \begin{minipage}{\marginparwidth}
%    \centering
%    \includegraphics[width=0.9\marginparwidth]{figures/cats}
%    \caption{In this image, the cats are tessellated within a square
%      frame. Images should also have captions and be within the
%      boundaries of the sidebar on page~\pageref{sec:sidebar}. Photo:
%      \cczero~jofish on Flickr.}~\label{fig:marginfig}
%  \end{minipage}
%\end{marginfigure}

%\begin{figure*}
%  \centering
%  \includegraphics[width=1.3\columnwidth]{figures/map}
%  \caption{In this image, the map maximizes use of space. You can make
%    figures as wide as you need, up to a maximum of the full width of
%    both columns. Note that \LaTeX\ tends to render large figures on a
%    dedicated page. Image: \ccbynd~ayman on Flickr.}~\label{fig:cats}
%\end{figure*}



\balance{} 

\bibliographystyle{SIGCHI-Reference-Format}
\bibliography{MR_Concept}

\end{document}

%%% Local Variables:
%%% mode: latex
%%% TeX-master: t
%%% End:
